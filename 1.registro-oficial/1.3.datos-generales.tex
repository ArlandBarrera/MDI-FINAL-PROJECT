\\
\textbf{Estimado(a) Vicedecano(a):}

Por este medio le informamos que el Título del Trabajo de Graduación que he (hemos) escogido es:

\begin{tabularx}{\textwidth}{|X|}
  \hline
  Desarrollo de Prototipo de Simulación Computacional de las Fases de la Luna: Una Herrameinta Interactiva para  la Educación y Divulgación Astronómica. \\
  \hline
\end{tabularx}

EL mismo lleva como Objetivo General:

Desarrollar un prototipo de simulación computacional interactivo que represente de manera precisa las fases de la Luna, con el fin de mejorar la comprensión del fenómeno y promover la educación y divulgación astronómica.

El tema escogido tiene mayor relevancia en el área académica de: (ponderar en rangos de 5 a 1, donde 5 es para el departamento de mayor afinidad y 1 para el depatartamento de menor afinidad)

\begin{tabular}{|c|p{20em}|c|p{18em}|}
  \hline
  1 & Arquitectura y Redes de Computacionales & 5 & Computación y Simulación de Sistemas \\
  \hline
  5 & Ingeniería de Software & 4 & Programación de Computadoras \\
  \hline
  3 & \multicolumn{3}{l|}{\raggedright Sistemas de Información, Control y Evalación de Recursos Informáticos} \\
  \hline
\end{tabular}

Este trabajo de graduación lo consideramos de tipo:

\begin{tabular}{|p{7em}|c|p{5em}|c|p{5em}|c|p{8em}|c|p{6em}|c|}
  \hline
  Teórico & & Teórico Práctico & ✔ & Práctica Profesional & & Certificación & & Otro & \\
  \hline
  \multicolumn{3}{|l|}{Si es Otro, especifique:} & \multicolumn{7}{c|}{} \\
  \hline
\end{tabular}

Si es Práctica Profesional, Nombre de la Empresa:

\hspace*{2em} Para Optar por el Título de Licenciaura en: \small{Licenciaura en Dessarrollo de Software.}\normalsize\\
\hspace*{2em} \underline{X}\\
Sugerimos como Asesor al Profesor(a): Edmanuel Cruz

El cual pertenece al Departamento Académico:\\
\underline{Departamento de Computación y Simulación de Sistemas}

\begin{tabularx}{\textwidth}{
  | >{\raggedright\arraybackslash}X
  | >{\raggedright\arraybackslash}X
  | >{\raggedright\arraybackslash}X
  | >{\raggedright\arraybackslash}X
  | >{\raggedright\arraybackslash}X |}
  \hline
  \textbf{NOMBRE} & \textbf{CÉDULA} & \textbf{TELÉFONOS} & \textbf{CORREO} & \textbf{FIRMA} \\
  \hline
  & & & @utp.ac.pa & \\
  \hline
\end{tabularx}
\\
\itshape \small
Artículo 40 del Reglamento aprobado por el Consejo Académico: La Tesis será preferiblemente obra de un solo estudiante, pero por razones espaciales se permitirá más de un (1) estudiante es una misma Tesis.
\upshape \normalsize

% Columnas personalizadas b(big), s(small), t(tiny). Las 3 suman 100% ancho
\newcolumntype{b}{>{\hsize=.45\hsize}X} % ~45% ancho pagina
\newcolumntype{s}{>{\hsize=.3\hsize}X} % ~30% ancho pagina
\newcolumntype{t}{>{\hsize=.25\hsize}X} % ~25% ancho pagina
\begin{tabularx}{\textwidth}{|s|b|t|}
  \hline
  Fecha: Nº\underline{X} & Fecha:\underline{X} & \\
  \hline
  \textbf{Vo. Bo. Prof. Asesor} & \textbf{Vo. Bo. Vicedecano(a) Académico} & \textbf{Vo. Bo. Decano} \\
  \hline
\end{tabularx}
