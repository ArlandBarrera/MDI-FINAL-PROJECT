
Sin embargo, la mayoría de estas herramientas requieren hardware especializado o no permiten la consulta de datos astronómicos reales. 
Como respuesta a esta brecha, han surgido iniciativas basadas en software que utilizan cálculos precisos para predecir las fases de la Luna
en un intervalo de tiempo determinado, integrando modelos matemáticos y datos astronómicos provenientes de efemérides.

Estos avances han abierto nuevas posibilidades para la creación de recursos didácticos accesibles, que combinan precisión científica con representaciones visuales intuitivas.