Se plantea uan metodología investigativa para el desarrollo del prototipo que se estructura en varias fases, permitiendo una ejecución
sistemática y coherente que garantice el cumplimiento de los objetivos del proyecto. En primer lugar, se realizará 
una fase de investigación y análisis en la que se recopilará información sobre los fundamentos astronómicos de las
fases de la Luna, consultando fuentes académicas, libros especializados y bases de datos científicas. Se identificarán
las variables clave que afectan la visualización de las fases y se definirán los parámetros a simular.

Posteriormente, se procederá a la fase de diseño del prototipo que abarcará la arquitectura
del software y la interfaz de usuario. En esta etapa se seleccionarán las herramientas y lenguajes de programación adecuados
(por ejemplo, Python y JavaScript ) que permitan la integración de gráficos interactivos y simulaciones en tiempo real.
Además, se definirán las interacciones que el usuario podrá realizar, como manipular el ángulo de visión o ajustar la posición
de la Tierra y la Luna.

La fase de desarrollo incluirá la codificación y la integración de módulos, combinando algoritmos matemáticos con elementos gráficos.
Se implementarán pruebas unitarias y de integración para garantizar la precisión y estabilidad del prototipo. Durante este proceso se
fomentará la documentación continua del código y se establecerán mecanismos de control de versiones.

Finalmente, se realizará una fase de validación y retroalimentación en la que el prototipo será de código abierto para ser
evaluados  por expertos en astronomía y educación ya sea a nivel nacional o internacional.