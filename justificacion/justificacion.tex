La propuesta se fundamenta en la necesidad de innovar los métodos de enseñanza de la astronomía, abordando esa existente entre la teoría y la práctica. El uso de simulaciones interactivas permite a los usuarios experimentar de manera directa con los conceptos astronómicos, facilitando el aprendizaje mediante la visualización y manipulación de variables en tiempo real. Esto no solo mejora la comprensión del fenómeno de las fases de la Luna, sino que también fomenta un aprendizaje activo y participativo.

Además, el desarrollo de un prototipo basado en software de código abierto garantiza una solución de bajo costo y alta accesibilidad, lo cual es especialmente relevante para instituciones educativas con recursos limitados. 